\documentclass[a4paper, 12pt]{article}
\usepackage[backend=biber]{biblatex}
\addbibresource{ethics-essay.bib}

\title{Bioinformatics and Bioimaging: Considerations for the 21^{st} Century}
\author{Michael J. Jones \\
  MSc Bioinformatics \& Computational Genomics \\
  Queen's University Belfast}

\date{\today}
\begin{document}

\section{Introduction}
In 2006, Professor Herman Tavani wrote that both Information Technology and
genetics will play a critical role in $21^{st}$ century healthcare. As part of
this rapid development, Bioinformatics as a field is exemplary of how
Information Technology is required to collate, analyse and distribute genetic
data \cite{tavani2006ethics}[Chp.~17].

The abundance of data available in the form of genetic and image data raises
concerns for privacy of individuals. The fact that Informatics has come to
play such a critical role in biomedical research poses problems ethically.
Fields such as Bioethics and Computational ethics have had years to develop
independently.

Hongladarom argues that Bioinformatics poses a whole new paradigm of ethics in
terms of data protection. While data protection measures are in place for the
safe keeping of credit card numbers in order to protect the identity of an
individual, storing a persons full genome or exome sequence adds an additional
challenge in that your genetic sequence is inherently and indelibly part of
your identity \cite{hongladarom2006ethics}.

This article will expand on some of the ethical challenges that come with
Bioimaging and Bioinformatics with particular attention given to the storing
of data and tissue in Biobanks, followed by a discussion of the ethics of
image manipulation in a scientific context.

\section{Biobanks}
An important element for successful scientific enquire in the field of
Bioinformatics is the availability of large data sources for processing. The
term `biobank` is a confounding term resulting from a range of different
interpretations, be they the storage of tissue samples, DNA sequence data or
medical images.

An early and comprehensive example of a Biobank took blood samples from the
many Icelandic citizens including data from their medical records. deCODE
genetics is a private biotechnology company from Delaware that was allowed to
gain access to these records following a law passed in the Icelandic parliament
in 1998 \cite{chadwick1999icelandic}.

A number of ethical issues have been raised following the decision of
the Icelandic parliament. These issues will be discussed below.

\subsection{Informed consent}
A large proponent of the criticism of the deCODE genetics project is the issue
of informed consent with regard to gaining access to individuals private
medical data.

\section{Bioimaging}

\section{Discussion}

\printbibliography
\end{document}
